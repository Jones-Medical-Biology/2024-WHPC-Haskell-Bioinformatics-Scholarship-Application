\documentclass[10pt,letterpaper]{letter}
\usepackage{hyperref}
\usepackage[utf8]{inputenc}

\signature{Noah Thomas Jones}
\address{1200 Newell Dr. \\ R1-144 Academic Research Building \\ University of Florida \\ Gainesville, FL 32610}

\begin{document}

\begin{letter}{Assistant Professor Emma ``Mickey'' MacKey \\ 221 Williamson Hall
\\ University of Florida \\ Gainesville, FL 32603}
\opening{Dear Dr. MacKey,}

I, Noah Thomas Jones, intend to serve as a faculty mentor to oversee the enclosed project of Calypsa McCarthy and Salma Ouaakki. Miss McCarthy is a biology major in IFAS, and Miss Ouaakki is a biomedical engineering major in the College of Engineering. Calypsa McCarthy has been volunteering for the past few months studying basic lab techniques and learning foundational skills in bioinformatics, and Miss Ouaakki has a solid background in the basics of programming in C++. Their academic foundations are adequate for the project that they are proposing.

Together, these two bright and ambitious scholars sought me out because of their overlapping research goals. We developed a research proposal around the concept of writing a new core framework for bioinformatics with correctness and scalability in mind. Their proposal is both within their zone of proximal development as young academics but also has the chance to make a real impact by expanding the available bioinformatics tool set in particular with respect to the sorts of data that impact our research.

In the course of completing this project, they will have to overcome hurdles of increasing intellectual difficulty: they will produce a statically typed parser that will improve the parsing of genetic sequence and expression data. Next, they will replicate the functionality of common dimensionality-reduction tools and write tests to demonstrate the useful properties that Haskell brings to the table with respect to forcing conformation to data standards and improving verification and accuracy. Furthermore, they aim to improve upon existing methods by integrating with parallelism that functionality developed in the previous phase, and then finally, should all other aims succeed, their ambitious aim AI tools coded in C++.

To successfully complete these milestones, they will need to review the applicable literature to develop expertise on contemporary developments in bioinformatics as well statically typed approaches to bioinformatics. They will develop intermediate skills programming in Haskell and become familiar with industry standard practices. They will also need to deepen their knowledge of the datasets that we will be using. To this end, they will be assisted by a graduate student, who will also be developing a Haskell seminar series under the auspices of the Practicum AI training curriculum, the aim being to build on synergies when coding with other bioinformaticians who are interested in using Haskell in their research projects.

The scope of impact of this project will encompass both the the Haskell community and those engaged in our similar avenues to our own. Bioinformatics libraries in Haskell are presently sparse and underdeveloped, so the work being performed by these students will be a significant contribution to the Haskell community. Furthermore, there are major challenges facing in the bioinformatics community, and these have been well articulated by the students' background statement. 

I am fortunate to have such exceptional students, and I cannot recommend Miss McCarthy and Miss Ouaakki and their project highly enough.

\closing{Yours Faithfully,}


\end{letter}
\end{document}

